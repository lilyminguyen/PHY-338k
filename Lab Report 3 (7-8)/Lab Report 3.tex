\documentclass{article}
\usepackage{graphicx}
\usepackage{amsmath}
\usepackage{float}
\usepackage{siunitx}
\usepackage{caption}
\usepackage{subcaption}
\usepackage{hyperref}

\title{PHY 338k Lab Report 3}
\author{Lily Nguyen with Anna Grove}
\date{October 9, 2025}

\begin{document}

\maketitle

\section{Introduction}


\section{Lab 7: Operational Amplifiers I}

\subsection{Open-loop op-amp, comparator}

\textit{include copy of scope display of $V_\text{out}$ vs. $V_\text{in}$}\\

\noindent\textit{explain why linear gain region of op-amp (slope near $V_\text{in}=0$) does not play a role in this circuit}\\

\noindent\textit{measure max and min output voltage, are these the same or diff than the supply voltages?}\\

\noindent\textit{compare what you get to the maximum voltage swing shown on the spec sheet}\\

\noindent\textit{measure the slew rate (clope of rise or fall of output waveform)}\\



\subsection{Inverting amplifier}

\textit{plot gain vs. f for both resistance ratios together on the same log-log plot}\\

\noindent\textit{extrapolate from the plot curves (can use hand ruler) to get a rough estimate of the open-loop gain as a function of frequency}\\

\noindent\textit{use the 2 rules for an op-amp (given in manual) to explain/derive how the gain is produced for the inverting amplifier}



\subsection{Summing amplifier}

\textit{report each voltage and resistor combination used along with the corresponding output voltage for each}\\

\noindent\textit{use what you observed to determine an equation relating the input and output voltages to the 3 resistor values. Generalize this equation for $N$ inputs}




\section{Lab 8: Operational Amplifiers II}



\section*{Acknowledgements}

We would like to thank our laboratory instructor, Dr. Heinzen, and our teaching 
assistant, Kyle Gable, for their guidance and support through these labs.Their 
assistance and feedback were invaluable in helping us understand the experimental 
procedures and circuit concepts. We also appreciate the department's laboratory 
facilities and equipment, which made these experiments possible.

\end{document}